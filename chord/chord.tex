\def\dicover{V496}

\usepackage{tikz} % dessin
\usetikzlibrary{arrows.meta,bending,shapes.geometric,shapes.misc,positioning}
\usepackage{xparse} % for NewDocumentCommand

%%%%
\input{chord/chparse.tex}
%%%%


%%%%%%
% Chord
%%%%%%

% Usage \ST{B}{M}{H}[scale] - draw a stac 3 notes
\NewDocumentCommand\ST{mmm O{1.0}}{\scalebox{#4}{\tikz[baseline=0.ex]{
    \begin{scope}[node distance=0.4ex,inner sep=0pt,node font=\Large]
        \node[anchor=base] (M) at (0,0) {#2};
        \node [above=of M,anchor=south] (T) {#3};
        \node [below=of M,anchor=north] (B) {#1};
        
        \node [above=of T,anchor=south] (H) {};% extra space at TOP
        \node [below=of B,anchor=north] (L) {};% extra space at BOTTOM

    \end{scope}
    }}}

% usage: \LS{note F}[scale] - Lateral symbol of the Fundamental decoration
\newdimen\LSsize
\LSsize=1ex
\NewDocumentCommand\LS{O{1.0}}{\scalebox{#1}{\tikz[baseline=(B.base)]{
    \node (T) at (0,\LSsize) {};
    \node (B) at (0,-\LSsize) {};
    \path[draw=black,line width=.25pt](T)--(B);
    }}}

% usage: \FU{note F}[scale] - Fundamental decoration 
\NewDocumentCommand\FU{m O{1.0}}{\LS[#2]#1\LS[#2]}


\ExplSyntaxOn
%Usage:  \CH{note F}{note T}{note Q}{inversion}[scale] - draw the right Stack depending of inversion F/T/Q
\NewDocumentCommand\CH{mmmm O{1.0}}
{
	  \tl_set:Ne \l__Inversion_tl {#4} % fill by Cadence
       
       % Trace
		\iow_term:n {CH}
		\tl_show:N \l__Inversion_tl

	\str_case:NnTF #4 % what Inversion
	{
		{F}
			{
				\iow_term:n {CH-F}
				\ST{\FU{#1}[#5]}{#2}{#3}
			}
		{T}
			{
				\iow_term:n {CH-T}
				\ST{#2}{#3}{\FU{#1}[#5]}
			}
		{Q}
			{
				\iow_term:n {CH-Q}
				\ST{#3}{\FU{#1}[#5]}{#2}
			}
	}
	{% additional things if found
	}
	{% additional things if other
	}
		
}
\ExplSyntaxOff

% usage: \CE{color}[scale] - Empty in chord
%\NewDocumentCommand\CE{m}{\color{#1}{X}}
\newdimen\CEsize
\CEsize=7pt
\newdimen\CEwidth
\CEwidth=0.2pt
\NewDocumentCommand\CE{m O{1.0}}{\scalebox{#2}{\tikz[baseline=-\CEsize]{
    \path[draw=#1,line width=\CEwidth,-] (0,-\CEsize)--(\CEsize,\CEsize);
    \path[draw=#1,line width=\CEwidth,-] (0,\CEsize)--(\CEsize,-\CEsize);
    }}}

\newdimen\ARPArrowSize
\ARPArrowSize=3ex
\newdimen\ARPArrowBAseLine
\ARPArrowBAseLine=-1ex
% usage: \UAC[color][scale] - Up Arrow for arpege
\NewDocumentCommand\UAC{O{black} O{1.0}}{\scalebox{#2}{\tikz[baseline=-\ARPArrowSize/2,,>={Stealth}]{
    \path[draw=#1,line width=1.5pt,->](0ex,-\ARPArrowSize+\ARPArrowBAseLine)--(0ex,\ARPArrowSize+\ARPArrowBAseLine);
    }}}
% usage: \DAC[color][scale] - Down Arrow for arpege
\NewDocumentCommand\DAC{O{black} O{1.0}}{\scalebox{#2}{\tikz[baseline=-\ARPArrowSize/2,>={Stealth}]{\path[draw=#1,line width=1.5pt,->](0ex,\ARPArrowSize+\ARPArrowBAseLine)--(0ex,-\ARPArrowSize+\ARPArrowBAseLine);}}}

% usage: \UCA{note F}{note T}{note Q}{F or T or Q}[scale]  - Arpege UP 
\NewDocumentCommand\UCA{mmmm O{1.0}}{\UAC\CH{#1}{#2}{#3}{#4}[#5]}
% usage: \DCA{note F}{note T}{note Q}[F or T or Q][scale]  - Arpege down 
\NewDocumentCommand\DCA{mmmm O{1.0}}{\DAC\CH{#1}{#2}{#3}{#4}[#5]}

%%%%%%
% Cadence
%%%%%%

% usage: \Cadence{Chord ... Chord}[scale]  - Several Chords in a string 
\ExplSyntaxOn
\NewDocumentCommand\Cadence{m O{1.0}}
{
    \ParseCadence{#1}
    
    
    % Parse each Chord
    \seq_map_indexed_inline:Nn \l__Cadence_seq   
    {
        \tl_set:Ne \l__CleanF_tl {\seq_item:Nn \l__CleanF_seq {##1}}
        \tl_set:Ne \l__CleanT_tl {\seq_item:Nn \l__CleanT_seq {##1}}
        \tl_set:Ne \l__CleanQ_tl {\seq_item:Nn \l__CleanQ_seq {##1}}
        
        \str_set:Ne \l__MuteF_bool_str {\seq_item:Nn \l__MuteF_bool_seq {##1}}
        \str_set:Ne \l__MuteT_bool_str {\seq_item:Nn \l__MuteT_bool_seq {##1}}
        \str_set:Ne \l__MuteQ_bool_str {\seq_item:Nn \l__MuteQ_bool_seq {##1}}
        
        \tl_set:Ne \l__Inversion_tl {\seq_item:Nn \l__Inversion_seq {##1}}
    
        \tl_set:Ne \l__Arpege_tl {\seq_item:Nn \l__Arpege_seq {##1}}
        \tl_set:Ne \l__Arpege_bool_str {\seq_item:Nn \l__Arpege_bool_seq {##1}}
        
        % constante in fact to test bool
        \str_set:Nn \l__true_str {true}
    
        % create the 3 Notes (Muted or not Muted)
        \str_if_eq:NNTF \l__MuteF_bool_str \l__true_str
            {% Generate the XClean Note
            \tl_set_eq:NN \l__F_tl \l__CleanF_tl
            \tl_put_left:Nn \l__F_tl {dcX}
                % If not exist replace by the unknown note
                \cs_if_exist:cTF {\l__F_tl}
                {}%ok
                {\tl_set:Nn \l__F_tl {dcXUKw}}                
            }
            {% Generate the Clean note
            \tl_set_eq:NN \l__F_tl \l__CleanF_tl
            \tl_put_left:Nn \l__F_tl {dc} 
                % If not exist replace by the unknown note
                \cs_if_exist:cTF {\l__F_tl}
                {}%ok
                {\tl_set:Nn \l__F_tl {dcUKw}}                
            }
        \str_if_eq:NNTF \l__MuteT_bool_str \l__true_str
            {% Generate the XClean Note
            \tl_set_eq:NN \l__T_tl \l__CleanT_tl
            \tl_put_left:Nn \l__T_tl {dcX} 
                % If not exist replace by the unknown note
                \cs_if_exist:cTF {\l__T_tl}
                {}%ok
                {\tl_set:Nn \l__T_tl {dcXUKw}}                
            }
            {% Generate the Clean note
            \tl_set_eq:NN \l__T_tl \l__CleanT_tl
            \tl_put_left:Nn \l__T_tl {dc} 
                % If not exist replace by the unknown note
                \cs_if_exist:cTF {\l__T_tl}
                {}%ok
                {\tl_set:Nn \l__T_tl {dcUKw}}                
            }
        \str_if_eq:NNTF \l__MuteQ_bool_str \l__true_str
            {% Generate the XClean Note
            \tl_set_eq:NN \l__Q_tl \l__CleanQ_tl
            \tl_put_left:Nn \l__Q_tl {dcX} 
                % If not exist replace by the unknown note
                \cs_if_exist:cTF {\l__Q_tl}
                {}%ok
                {\tl_set:Nn \l__Q_tl {dcXUKw}}
            }
            {% Generate the Clean note
            \tl_set_eq:NN \l__Q_tl \l__CleanQ_tl
            \tl_put_left:Nn \l__Q_tl {dc} 
                % If not exist replace by the unknown note
                \cs_if_exist:cTF {\l__Q_tl}
                {}%ok
                {\tl_set:Nn \l__Q_tl {dcUKw}}
            }
        
        % Draw \CE, \DCA or \UCA
        \str_if_eq:NNTF \l__Arpege_bool_str \l__true_str
        {% Yes Arpege
            \str_set:Nn \D {D} 
            \str_if_eq:NNTF \l__Arpege_tl \D
            { % Down
                \DCA{\use:c {\l__F_tl}}{\use:c {\l__T_tl}}{\use:c {\l__Q_tl}}{\l__Inversion_tl}[#2] \quad             
            }
            { % UP
                \UCA{\use:c {\l__F_tl}}{\use:c {\l__T_tl}}{\use:c {\l__Q_tl}}{\l__Inversion_tl}[#2] \quad
            }
        }
        {% NO Arpege
            \CH{\use:c {\l__F_tl}}{\use:c {\l__T_tl}}{\use:c {\l__Q_tl}}{\l__Inversion_tl}[#2] \quad
        }    
    }
}
\ExplSyntaxOff

