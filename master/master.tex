\def\mastver{V200}

\usepackage{tikz}
\usetikzlibrary{bending,shapes.geometric,shapes.misc,positioning,decorations.pathmorphing}
\usepgfmodule{oo}

%%%%
\def\mspaver{V101}


\ExplSyntaxOn
% Parse Master and push the result in variable
    % Extract R/L
    % Prepare seq(s)
    % Parse the notes
% #1 = Master
\NewDocumentCommand\MSTParseMaster{m}
{
    % Temp variables
    \tl_set:Ne \l__Master_tl {#1} % fill by Cadence
    
    % Result variables at the Master level
    \seq_clear_new:N \l__Master_seq % raw Master seq
    \bool_set_false:N \l__Left_bool % true if start with a left hand

    % Result variables at the Notes level
    \seq_clear_new:N \l__Note_seq
    \seq_clear_new:N \l__NoteScale_seq

         % Trace
        \iow_term:n {MSTParseMaster}
        \tl_show:N \l__Master_tl

    % Parse R/L
    \MstParseRL \l__Master_tl \l__Left_bool

         % Trace
        \iow_term:n {ParseMaster-RL}
        \tl_show:N \l__Master_tl
        \bool_show:n \l__Left_bool

    
    % Parse each Note
    % Split the Notes on space
    \seq_set_split:NnV \l__Master_seq {~} \l__Master_tl
        % Trace
        \iow_term:n {ParseMaster-split}
        \seq_show:N \l__Master_seq

    % Parse each Note
    \seq_map_inline:Nn \l__Master_seq
    {
        \MstParseNote{##1}
        \seq_put_right:NV \l__Note_seq \l__Note_tl
        \seq_put_right:NV \l__NoteScale_seq \l__NoteScale_tl
    }
        % Trace
        \iow_term:n {ParseMaster-end}
        \seq_show:N \l__Master_seq
        \seq_show:N \l__Note_seq
        \seq_show:N \l__NoteScale_seq

}

% Parse Master and Detect R or L
% #1 = Master
% #2 = Boolean true if left
\NewDocumentCommand\MstParseRL{mm}
{
    % Temp variables
    \tl_set:Ne \l__MstTmp_tl {#1} % fill by Master to be parsed
    \seq_clear_new:N \l__MstTmp_seq
    
    % extract [R/L]
    \regex_extract_once:nVN {\[(.*?)\]} \l__MstTmp_tl \l__MstTmp_seq

        % Trace
        \iow_term:n {MstParseRL}
        \seq_show:N \l__MstTmp_seq
    
    \int_compare:nNnTF {\seq_count:N \l__MstTmp_seq} > {1} % test if found
        {% yes found
                % Trace
                \iow_term:n {MstParseRL-Found} 
            
            \bool_set_true:N #2            
            % delete R/L +  no space
            \regex_replace_once:nnN {\[.*?\]} {} \l__MstTmp_tl
        }
        {% no R by default
                % Trace
                \iow_term:n {MstParseRL-not Found}
                
            \bool_set_false:N #2
        }

        % Trim space 
        \tl_set:Ne \l__MstTmp_tl {\tl_trim_spaces:V \l__MstTmp_tl}
        % store result
        \tl_set_eq:NN #1 \l__MstTmp_tl
        
        % Trace
        \iow_term:n {MstParseRL-end}
        \tl_show:N \l__MstTmp_tl
        \tl_show:N #1
        \bool_show:N #2
}

% Parse Note and detect musical scale
% #1 = Notes
\NewDocumentCommand\MstParseNote{m}
{
    % Temp variables
    \tl_set:Ne \l__NoteTmp_tl {#1}
    \tl_set:Ne \l__ScaleTmp_tl {#1}

        % Trace
        \iow_term:n {MstParseNote}
        \tl_show:N \l__NoteTmp_tl
        
    % Result variables at the Notes level
    \tl_clear_new:N \l__Note_tl
    \tl_clear_new:N \l__NoteScale_tl    
    
    % do nothing if void
    \tl_if_empty:NTF \l__NoteTmp_tl
    {%void
    }
    {%Not Void
       % Extract the last char for the Scale
        \tl_reverse:N \l__ScaleTmp_tl
        \tl_set:Ne \l__NoteScale_tl {\tl_head:N \l__ScaleTmp_tl}
    
        % Clean the Note but find the last one
        % to correct the MiM BUG
        \tl_reverse:N \l__NoteTmp_tl
        \tl_remove_once:NV \l__NoteTmp_tl \l__NoteScale_tl
        \tl_reverse:N \l__NoteTmp_tl
        
        \tl_set_eq:NN \l__Note_tl \l__NoteTmp_tl
    }
        % Trace
        \iow_term:n {MstParseNote-end}
        \tl_show:N \l__Note_tl
        \tl_show:N \l__NoteScale_tl

}

\ExplSyntaxOff
   
%%%%


\pgfooclass{Symbol}{
    % This is the class Note
    \attribute Angle;
    \attribute Radius; % for the ding
    \attribute Color;
    
    % The constructor
    \method Symbol(#1,#2,#3) { 
      \pgfooset{Angle}{#1}
      \pgfooset{Radius}{#2}
      \pgfooset{Color}{#3}
    }
    
    % Get the color {Result} 
    \method GetColor(#1) {
        \pgfooget{Color}{#1}
    }

    % Get the Radius {Result} 
    \method GetRadius(#1) {
        \pgfooget{Radius}{#1}
    }
        
     % Get the Angle {Result} 
    \method GetAngle(#1) {
        \pgfooget{Angle}{#1}
    }

}
% create Notes
\pgfoonew \SymREd=new Symbol(0,0,re)
\pgfoonew \SymLAb=new Symbol(-90,1,la)
\pgfoonew \SymDoB=new Symbol(-45,1,do)
\pgfoonew \SymMIm=new Symbol(0,1,mi)
\pgfoonew \SymSOLu=new Symbol(36,1,sol)
\pgfoonew \SymDOu=new Symbol(72,1,do)
\pgfoonew \SymLAu=new Symbol(108,1,la)
\pgfoonew \SymFAu=new Symbol(144,1,fa)
\pgfoonew \SymREm=new Symbol(180,1,re)
\pgfoonew \SymSIb=new Symbol(225,1,si)

\pgfoonew \SymMIh=new Symbol(45,1,mi)
\pgfoonew \SymSOLh=new Symbol(72,1,sol)
\pgfoonew \SymLAh=new Symbol(96,1,la)
\pgfoonew \SymFAh=new Symbol(120,1,fa)
\pgfoonew \SymREh=new Symbol(144,1,re)

\pgfoonew \SymMIl=new Symbol(-22.5,1,mi)
\pgfoonew \SymSOLl=new Symbol(225,1,sol)
\pgfoonew \SymFAl=new Symbol(180,1,fa)

\pgfoonew \SymGOq=new Symbol(202.5,1,go)
\pgfoonew \SymGOs=new Symbol(-22.5,1,go)

\pgfoonew \SymUKw=new Symbol(0,0,uk)
   

% Define Figure scale
\newdimen\Notesize
\Notesize=22pt
% define scale
\newdimen\Vscale
\Vscale=16pt
\newdimen\Vzero
\Vzero=51pt
\newdimen\Hscale
\Hscale=40pt
% ligne scale
\newdimen\LineS
\LineS=-40pt
\newdimen\LineA
\LineA=80pt

% define Beat-oo structure
\pgfooclass{Beat}{
    % This is the class Note
    \attribute Xpos;
    \attribute Ypos;
    \attribute Color;
    \attribute LineColor;
    
    % The constructor
    \method Beat(#1,#2,#3,#4) { 
      \pgfooset{Xpos}{#1*\Hscale}
      \pgfooset{Ypos}{-#2*\Vscale+\Vzero}
      \pgfooset{Color}{#3}
      \pgfooset{LineColor}{#4}
    }
    
    % Get the color {Result} 
    \method GetColor(#1) {
        \pgfooget{Color}{#1}
    }
        
    % Draw the Beat:
    % #1 D=Draw C=Fill by Color F= figure (Scale or Ghost) 
    % #2 color or Pic or angle
    % #3 radius for the symbol
    \method DrawBeat(#1,#2,#3) {
        
        \IfEqCase{#1} % what is requested ?
        {
        {D} % draw
            {
                \node[circle,draw=\pgfoovalueof{Color},line width=.5pt,inner sep=0.45*\Notesize] (A) at (\pgfoovalueof{Xpos},\pgfoovalueof{Ypos}) {};
            }
        {C} % fill color =#2
            {
                \node[circle,fill=#2,inner sep=0.45*\Notesize] (A) at (\pgfoovalueof{Xpos},\pgfoovalueof{Ypos}) {};
            }    
        {F} % Figure Pic= #2
            {
                \pic at (\pgfoovalueof{Xpos},\pgfoovalueof{Ypos}) {#2};
            }
        {S} % Symbol rotate = #2,#3
            {
                \pic at (\pgfoovalueof{Xpos},\pgfoovalueof{Ypos})  [rotate=#2] {Symb=#3*\Notesize/2};
            } 
        {L} % time line
            {
                \node (S) at (\LineS,\pgfoovalueof{Ypos}) {};
                \node (A) at (\LineA,\pgfoovalueof{Ypos}) {};
                \path [draw=\pgfoovalueof{LineColor}] (S) -- (A); 
            }
        }
    }
}

% create each Beat (up to 8) with R/L
% Right first
\pgfoonew \RightA=new Beat(1,0,right,black)
\pgfoonew \RightB=new Beat(0,1,left,white)
\pgfoonew \RightC=new Beat(1,2,right,gray)
\pgfoonew \RightD=new Beat(0,3,left,white)
\pgfoonew \RightE=new Beat(1,4,right,black)
\pgfoonew \RightF=new Beat(0,5,left,white)
\pgfoonew \RightG=new Beat(1,6,right,gray)
\pgfoonew \RightH=new Beat(0,7,left,white)

% Left first
\pgfoonew \LeftA=new Beat(0,0,left,black)
\pgfoonew \LeftB=new Beat(1,1,right,white)
\pgfoonew \LeftC=new Beat(0,2,left,gray)
\pgfoonew \LeftD=new Beat(1,3,right,white)
\pgfoonew \LeftE=new Beat(0,4,left,black)
\pgfoonew \LeftF=new Beat(1,5,right,white)
\pgfoonew \LeftG=new Beat(0,6,left,gray)
\pgfoonew \LeftH=new Beat(1,7,right,white)

\ExplSyntaxOn
% create a seq of Beats (R/L)
\seq_clear_new:N \l__RightBeats_seq
\seq_clear_new:N \l__LeftBeats_seq

\seq_put_right:Nn \l__RightBeats_seq {RightA}
\seq_put_right:Nn \l__RightBeats_seq {RightB}
\seq_put_right:Nn \l__RightBeats_seq {RightC}
\seq_put_right:Nn \l__RightBeats_seq {RightD}
\seq_put_right:Nn \l__RightBeats_seq {RightE}
\seq_put_right:Nn \l__RightBeats_seq {RightF}
\seq_put_right:Nn \l__RightBeats_seq {RightG}
\seq_put_right:Nn \l__RightBeats_seq {RightH}

\seq_put_right:Nn \l__LeftBeats_seq {LeftA}
\seq_put_right:Nn \l__LeftBeats_seq {LeftB}
\seq_put_right:Nn \l__LeftBeats_seq {LeftC}
\seq_put_right:Nn \l__LeftBeats_seq {LeftD}
\seq_put_right:Nn \l__LeftBeats_seq {LeftE}
\seq_put_right:Nn \l__LeftBeats_seq {LeftF}
\seq_put_right:Nn \l__LeftBeats_seq {LeftG}
\seq_put_right:Nn \l__LeftBeats_seq {LeftH}

\ExplSyntaxOff

% Draw beat from index + R/L
% Usage DrawBeatFromName: Draw from name {D=draw F=Filled ... }[Color]
% #1 Index
% #2 to draw
% #3 Color or angle
% #4 Radius
\ExplSyntaxOn
\NewDocumentCommand\DrawBeatFromSeq{mm O{} O{}}{
        % Trace
        \iow_term:n {DrawBeatFromSeq}
        \iow_term:n {#1}
        \iow_term:n {#2}
        \iow_term:n {#3}
        \iow_term:n {#4}

        
    % find BeatName
    \bool_if:NTF \l__Left_bool % true if left
        {% left
            \tl_set:Ne \l__BeatName_tl {\seq_item:Nn \l__LeftBeats_seq {#1}}
        }
        {% right        
            \tl_set:Ne \l__BeatName_tl {\seq_item:Nn \l__RightBeats_seq {#1}}
        }

        % Trace
        \iow_term:n {DrawBeatFromSeq-before-draw}
        \tl_show:N \l__BeatName_tl

    % Draw
    \use:c {\l__BeatName_tl}.DrawBeat(#2,#3,#4)
}
\ExplSyntaxOff

% symbol from Note Name and beat + R/L
% #1 Index
% #2 to draw
% #3 Pic to generate
\ExplSyntaxOn
\NewDocumentCommand\DrawFigureFromSeq{mmm}{
    % generate Pic
    \use:c {#3}
    % Draw
    \DrawBeatFromSeq{#1}{#2}[#3][]
}
\ExplSyntaxOff

% draw Overall Shape + R/L
\ExplSyntaxOn
\NewDocumentCommand\DrawMasterShape{}{

        % Trace
        \iow_term:n {DrawMasterShape}
 
    % for each note
    \seq_map_indexed_inline:Nn \l__Master_seq
    {
             % Trace
            \iow_term:n {DrawMasterShape-indexed}
            \iow_term:n {##1}
            \iow_term:n {##2}
                                  
       \DrawBeatFromSeq {##1} {L} % Draw Horizontal line 
       \DrawBeatFromSeq {##1} {D} % just draw the shape       
    }    
}
\ExplSyntaxOff

% draw Name



% Draw Notes 
\ExplSyntaxOn
\NewDocumentCommand\DrawNotes{}{

        % Trace
        \iow_term:n {DrawNotes}
 
    % for each note
    \seq_map_indexed_inline:Nn \l__Master_seq
    {
        % create the Symbol name
        \tl_clear_new:N \l__Symbol_tl
        \tl_set:Ne \l__Symbol_tl {##2}
        \tl_put_left:Nn \l__Symbol_tl {Sym}
        
        % get the color from Symbol
        \tl_clear_new:N \l__Color_tl
        % If not exist replace by the unkwow note
        \cs_if_exist:cTF{\l__Symbol_tl}
            {\use:c {\l__Symbol_tl}.GetColor(\l__Color_tl)}
            {\use:c {SymUKw}.GetColor(\l__Color_tl)}
        
        % get the symbol angle from the Symbol
        \tl_clear_new:N \l__Angle_tl
        % If not exist replace by the unkwow note
        \cs_if_exist:cTF{\l__Symbol_tl}
            {\use:c {\l__Symbol_tl}.GetAngle(\l__Angle_tl)}
            {\use:c {SymUKw}.GetAngle(\l__Angle_tl)}

        % get the symbol Radius from the Symbol
        \tl_clear_new:N \l__Radius_tl
        \cs_if_exist:cTF{\l__Symbol_tl}
            {\use:c {\l__Symbol_tl}.GetRadius(\l__Radius_tl)}
            {\use:c {SymUKw}.GetRadius(\l__Radius_tl)}
  
             % Trace
            \iow_term:n {DrawNotes-indexed}
            \iow_term:n {##1}
            \iow_term:n {##2}
            \tl_show:N\l__Symbol_tl
            \tl_show:N \l__Color_tl
            \tl_show:N\l__Angle_tl
            \tl_show:N\l__Radius_tl
                                  
        \DrawBeatFromSeq {##1} {C} [\l__Color_tl] % fill with Note color  
        
        \Symb
        \DrawBeatFromSeq {##1} {S} [\l__Angle_tl] [\l__Radius_tl]% draw the symbol 
    }    
}
\ExplSyntaxOff

% Draw Figure 
\ExplSyntaxOn
\NewDocumentCommand\DrawFigures{}{

        % Trace
        \iow_term:n {DrawFigures}
 
    % for each note
    \seq_map_indexed_inline:Nn \l__NoteScale_seq
    {
        \tl_clear_new:N \l__TheScale_tl
	    \tl_set:Ne \l__TheScale_tl {##2}
        \str_case:NnTF \l__TheScale_tl %what scale
        {        
        {d} % Ding
            {
            \Ding
            \DrawBeatFromSeq {##1} {F} [Ding]
            }
        {l} % Low
            {
            \Low
            \DrawBeatFromSeq {##1} {F} [Low]
            }
        {b} % Bottom
            {
            \Bottom
            \DrawBeatFromSeq {##1} {F} [Bottom]
            }
        {m} % Middle
            {
            \Middle
            \DrawBeatFromSeq {##1} {F} [Middle]
            }
        {u} % Upper
            {
            \Upper
            \DrawBeatFromSeq {##1} {F} [Upper]
            }
        {h} % High
            {
            \High
            \DrawBeatFromSeq {##1} {F} [High]
            }
        {q} % Left Ghost
            {
            \LGhost
            \DrawBeatFromSeq {##1} {F} [LGhost]
            }
        {s} % Right Ghost
            {
            \RGhost
            \DrawBeatFromSeq {##1} {F} [RGhost]
            }
        }
        {} % additionnal inst if found
        {} % additionnal inst if not found
    }    
}
\ExplSyntaxOff

% draw each symbol
% size
\newdimen\Msep
\sep=.05ex
\newdimen\Msize
\Msize=3pt
\newdimen\Mdepth
\Mdepth=3pt
\newdimen\Mshift
\Mshift=2pt
\newdimen\Mlinwd
\Mlinwd=2pt

\NewDocumentCommand\Ding{}{
\tikzset{
      Ding/.pic={
        \node[circle,inner sep=\Msize,draw=white,line width=\Mlinwd] at (0,0) {};
        }}
}
\NewDocumentCommand\Low{}{
\tikzset{
      Low/.pic={
        \node[inner sep=0] (D) at (-2*\Msize,-\Mshift){};
        \node[inner sep=0] (A) at (2*\Msize,-\Mshift){};
        \draw[draw=white,line width=\Mlinwd] (D) parabola [parabola height=-\Mdepth] (A);

        \node[inner sep=0] (E) at (-2*\Msize,-3*\Mshift){};
        \node[inner sep=0] (B) at (2*\Msize,-3*\Mshift){};
        \draw[draw=white,line width=\Mlinwd] (E) parabola [parabola height=-\Mdepth] (B);
        }}
}
\NewDocumentCommand\Bottom{}{
\tikzset{
      Bottom/.pic={
        \node[inner sep=0] (D) at (-2*\Msize,-\Mshift){};
        \node[inner sep=0] (A) at (2*\Msize,-\Mshift){};
        \draw[draw=white,line width=\Mlinwd] (D) parabola [parabola height=-\Mdepth] (A);
        }}
}
\NewDocumentCommand\Middle{}{
\tikzset{
      Middle/.pic={
        \node[inner sep=0] (D) at (-2*\Msize,0){};
        \node[inner sep=0] (A) at (2*\Msize,0){};
        \draw[draw=white,line width=\Mlinwd] (D) to (A);
        }}
}
\NewDocumentCommand\Upper{}{
\tikzset{
      Upper/.pic={
        \node[inner sep=0] (D) at (-2*\Msize,\Mshift){};
        \node[inner sep=0] (A) at (2*\Msize,\Mshift){};
        \draw[draw=white,line width=\Mlinwd] (D) parabola [parabola height=\Mdepth] (A);
        }}
}
\NewDocumentCommand\High{}{
\tikzset{
      High/.pic={
        \node[inner sep=0] (D) at (-2*\Msize,\Mshift){};
        \node[inner sep=0] (A) at (2*\Msize,\Mshift){};
        \draw[draw=white,line width=\Mlinwd] (D) parabola [parabola height=\Mdepth] (A);

        \node[inner sep=0] (E) at (-2*\Msize,3*\Mshift){};
        \node[inner sep=0] (B) at (2*\Msize,3*\Mshift){};
        \draw[draw=white,line width=\Mlinwd] (E) parabola [parabola height=\Mdepth] (B);
        }}
}
\NewDocumentCommand\LGhost{}{
\tikzset{
      LGhost/.pic={
        \fill[decoration={snake,amplitude=-1.5,segment length=\Msize*1.5}] [color=left] (-2*\Msize,\Mshift) arc [start angle=180, end angle=0, radius=2*\Msize] (2*\Msize,\Mshift) -- (2*\Msize,-4*\Mshift) decorate{ -- (-2*\Msize,-4*\Mshift)} -- (-2*\Msize,\Mshift);
        
        \fill[fill=white] (-4*\Msize/5,\Mshift) circle [radius=\Msize/2];
        \fill[fill=white] (4*\Msize/5,\Mshift) circle [radius=\Msize/2];
        }}
}
\NewDocumentCommand\RGhost{}{
\tikzset{
      RGhost/.pic={
        \fill[decoration={snake,amplitude=-1.5,segment length=\Msize*1.5}] [color=right] (-2*\Msize,\Mshift) arc [start angle=180, end angle=0, radius=2*\Msize] (2*\Msize,\Mshift) -- (2*\Msize,-4*\Mshift) decorate{ -- (-2*\Msize,-4*\Mshift)} -- (-2*\Msize,\Mshift);
        
        \fill[fill=white] (-4*\Msize/5,\Mshift) circle [radius=\Msize/2];
        \fill[fill=white] (4*\Msize/5,\Mshift) circle [radius=\Msize/2];
        }}
}
\NewDocumentCommand\Symb{}{
\tikzset{
    pics/Symb/.style = {
    code = {
        \node[circle,inner sep=1pt,fill=white] at (0:##1) {};
        }}}
        
}

% Master
    % Parse the Master (R/L) + Note + scale
    % draw Overall Shape
    % Draw name
    % Draw Notes
% \master{[L] up to 8xnotes}[scale]  L or R=(default)
\ExplSyntaxOn
\NewDocumentCommand\Master{m O{1.0}}{\scalebox{#2}{\tikz[baseline=0ex]{

    % Temp variables
    \int_zero_new:N \l__NoteNumber_int
     \tl_set:Ne \l__MasterTmp_tl {#1}
       
        % Trace
        \iow_term:n {Master}
        \iow_term:n {\l__MasterTmp_tl}
   
    % do nothing if void
    \tl_if_empty:NTF \l__MasterTmp_tl
    {%void
    }
    {%Not Void
        % Parse the Master
        \MSTParseMaster{\l__MasterTmp_tl}
    
        % draw Overall Shape    
        \DrawMasterShape
        
        % draw Notes
        \DrawNotes
    
        % draw Figures
        \DrawFigures     

    }
    

    
}}}
\ExplSyntaxOff





