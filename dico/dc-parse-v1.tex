\def\dcpaver{V150}

\ExplSyntaxOn

% Parse Cadence and push the result in variable
    % Extract Name and Link
    % Ease Post parsing
    % Extract Alternate
    % Prepare seq(s)
    % Parse the notes
% #1 = Cadence
\NewDocumentCommand\ParseCadence{m}
{
    % Temp variables
    \tl_set:Ne \l__Cadence_tl {#1} % fill by Cadence
    
    % Result variables at the Cadence level
    \seq_clear_new:N \l__Cadence_seq % raw chords seq

    % Result variables at the Chord level
    \seq_clear_new:N \l__CleanF_seq
    \seq_clear_new:N \l__CleanT_seq
    \seq_clear_new:N \l__CleanQ_seq

    \seq_clear_new:N \l__MuteF_bool_seq
    \seq_clear_new:N \l__MuteT_bool_seq
    \seq_clear_new:N \l__MuteQ_bool_seq

    \seq_clear_new:N \l__Inversion_seq

    \seq_clear_new:N \l__Arpege_seq
    \seq_clear_new:N \l__Arpege_bool_seq


     % Trace
    \iow_term:n {ParseCadence}
    \tl_show:N \l__Cadence_seq
   
    % Split the Chords
    \seq_set_split:NnV \l__Cadence_seq {~} \l__Cadence_tl
    % Trace
    \iow_term:n {ParseCadence-split}
    \seq_show:N \l__Cadence_seq

    % Parse each Chord
    \seq_map_inline:Nn \l__Cadence_seq 
    {
        \ParseChord{##1}
        \seq_put_right:NV \l__CleanF_seq \l__CleanF_tl
        \seq_put_right:NV \l__CleanT_seq \l__CleanT_tl
        \seq_put_right:NV \l__CleanQ_seq \l__CleanQ_tl
        
        \seq_put_right:Ne \l__MuteF_bool_seq {\bool_to_str:N \l__MuteF_bool}
        \seq_put_right:Ne \l__MuteT_bool_seq {\bool_to_str:N \l__MuteT_bool}
        \seq_put_right:Ne \l__MuteQ_bool_seq {\bool_to_str:N \l__MuteQ_bool}
        
        \seq_put_right:NV \l__Inversion_seq \l__Inversion_tl
        
        \seq_put_right:NV \l__Arpege_seq \l__Arpege_tl
        \seq_put_right:Ne \l__Arpege_bool_seq {\bool_to_str:N \l__Arpege_bool}
                
    }
    % Trace
    \iow_term:n {CleanF}
    \seq_show:N \l__CleanF_seq
    \seq_show:N \l__CleanT_seq
    \seq_show:N \l__CleanQ_seq
    \seq_show:N \l__MuteF_bool_seq
    \seq_show:N \l__MuteT_bool_seq
    \seq_show:N \l__MuteQ_bool_seq
    \seq_show:N \l__Inversion_seq
    \seq_show:N \l__Arpege_seq
    \seq_show:N \l__Arpege_bool_seq
    
}


% get a note by it rank
% #1 = note number
\NewExpandableDocumentCommand\GetChord{m}
{
    \seq_item:Nn \l__Cadence_seq  { #1 }
}

% Search for a separator if yes raise a given bool and store before and after the sep
% \dcparse_SplitandSearch:nnnNN
% #1 string to parse / change in the rst depending of A/B
% #2 separator
% #3 Search (B) Before or (A) After the separator
% #4 Boolean to indicate if found
% #5 String Found
\cs_set:Npn \dcparse_SplitandSearch:NnnNN #1#2#3#4#5
{
\seq_set_split:NnV \l__SeqTmp_seq {#2} #1 % split the String
    %Trace
    \tl_analysis_show:n {\seq_set_split:NnV \l__SeqTmp_seq {#2} #1}
    \tl_show:N \l__SeqTmp_seq
    
    \int_compare:nNnTF {\seq_count:N \l__SeqTmp_seq}={2} % test if x has been found
        {% yes found
            \bool_set_true:N #4
            \str_set:Nn \A {A} 
            \str_if_eq:NNTF #3 \A
                {% A
                    \tl_set:Ne #5 {\seq_item:Nn \l__SeqTmp_seq {2}}% found after the sep
                    \tl_set:Ne #1  {\seq_item:Nn \l__SeqTmp_seq {1}}% the rest is before
                }
                {%B
                    \tl_set:Ne #5 {\seq_item:Nn \l__SeqTmp_seq {1}}% found before the sep
                    \tl_set:Ne #1  {\seq_item:Nn \l__SeqTmp_seq {2}}% the rest is after
                }
        }
        {% no
        }   
}
\cs_generate_variant:Nn \dcparse_SplitandSearch:NnnNN { NVnNN }

% Detect Mute in Notes X before note
% #1 Note to Parse
% #2 Bool to raise if Muted
\cs_set:Npn \dcparse_DetectMute:NN #1#2
{
    \str_set:Ne \l__head_str {\tl_head:N #1}
    %Trace
    \iow_term:n {dcparse_DetectMute}
    \tl_show:N #1
    \str_show:N \l__head_str
 
    
    \str_if_eq:VnTF \l__head_str {X}
        {% yes Muted
            \bool_set_true:N #2
            \tl_remove_once:Nn #1 {X}
        }
        {% no
        }
}

% Search for 3 notes with = between 
% \dcparse_ExtractNotes:N
% #1 Chord to parse
\cs_set:Npn \dcparse_ExtractNotes:N #1
{
    \seq_set_split:NnV \l__SeqTmp_seq {=} #1 % split the chord
    \int_compare:nNnTF {\seq_count:N \l__SeqTmp_seq}={3} % test if x has been found
        {% yes found
        \tl_set:Ne \l__CleanF_tl  {\seq_item:Nn \l__SeqTmp_seq {1}} % F
        \dcparse_DetectMute:NN \l__CleanF_tl \l__MuteF_bool

        \tl_set:Ne \l__CleanT_tl  {\seq_item:Nn \l__SeqTmp_seq {2}}  % T
        \dcparse_DetectMute:NN \l__CleanT_tl \l__MuteT_bool

        \tl_set:Ne \l__CleanQ_tl  {\seq_item:Nn \l__SeqTmp_seq {3}} % Q
        \dcparse_DetectMute:NN \l__CleanQ_tl \l__MuteQ_bool
        }
        {% no
        }   
}

% Extract the Arpege after [] in the Chord
% #1 Chord to parse
% #2 Arpege Found
% #3 Bool
\cs_set:Npn \dcparse_ExtractArpege:NN #1#2#3
{
    % Temp variables
    \tl_set:Ne \l__Chord_tl {#1} % fill by Chord to be parsed
    \seq_clear_new:N \l__SeqTmp_seq

   
    % detect and extract Arpege
    \regex_extract_once:nVN {\]([A-Za-z])} \l__Chord_tl \l__SeqTmp_seq

    % Trace
    \iow_term:n {dcparse_ExtractArpege}
    \seq_show:N \l__SeqTmp_seq

    \int_compare:nNnTF {\seq_count:N \l__SeqTmp_seq}>{1} % test if found
        {% yes found
            \bool_set_true:N #3
            \tl_set:Ne #2 {\seq_item:Nn \l__SeqTmp_seq  2} % \0 in seq 1 and \1 in seq 2
            % delete Arpege
            \regex_replace_once:nnN {\]([A-Za-z])} {\]} \l__Chord_tl
            \tl_set_eq:NN #1 \l__Chord_tl
        }
        {% no
        }
        
    % Trace
    \iow_term:n {dcparse_ExtractArpege-end}
    \tl_show:N \l__Chord_tl
    \tl_show:N #2
    \bool_show:N #3
        
}

% Extract the Inversion between [] in the Chord
% #1 Chord to parse
% #2 Invertion Found
\cs_set:Npn \dcparse_ExtractInversion:NN #1#2
{
    % Temp variables
    \tl_set:Ne \l__Chord_tl {#1} % fill by Chord to be parsed
    \bool_set_false:N \l__InvFound_bool
    \seq_clear_new:N \l__SeqTmp_seq
    
    % extract Invertion
    \regex_extract_once:nVN {\[(.*?)\]} \l__Chord_tl \l__SeqTmp_seq

    % Trace
    \iow_term:n {dcparse_ExtractInversion}
    \seq_show:N \l__SeqTmp_seq

    \int_compare:nNnTF {\seq_count:N \l__SeqTmp_seq}>{1} % test if found
        {% yes found
            \bool_set_true:N \l__InvFound_bool
            \tl_set:Ne #2 {\seq_item:Nn \l__SeqTmp_seq  2} % \0 in seeq 1 and \1 in seq 2
            % delete alternate
            \regex_replace_once:nnN {\[.*?\]} {} \l__Chord_tl
            \tl_set_eq:NN #1 \l__Chord_tl
        }
        {% no F by default
        \tl_set:Nn #2 {F}
        }
        
    % Trace
    \iow_term:n {dcparse_ExtractInversion-end}
    \tl_show:N \l__Chord_tl
    \tl_show:N #2
    \bool_show:N \l__InvFound_bool
        
}

% Parse a Chord
% #1 = note text
\NewDocumentCommand\ParseChord{m}
{
    % Temp variables
    \seq_clear_new:N \l__SeqTmp_seq
    \tl_set:Ne \l__Chord_tl {#1} % fill by Note to be parsed
    % Trace
    \iow_term:n {ParseChord}
    \tl_show:N \l__Chord_tl


    % Result variables
    \tl_clear_new:N \l__CleanF_tl
    \tl_clear_new:N \l__CleanT_tl
    \tl_clear_new:N \l__CleanQ_tl
    
    \bool_set_false:N \l__MuteF_bool % true mute
    \bool_set_false:N \l__MuteT_bool % true mute
    \bool_set_false:N \l__MuteQ_bool % true mute

    \tl_clear_new:N \l__Inversion_tl

    \tl_clear_new:N \l__Arpege_tl
    \bool_set_false:N \l__Arpege_bool % true Arpege

     
    % collect Arpege [ ]
    \dcparse_ExtractArpege:NN \l__Chord_tl \l__Arpege_tl \l__Arpege_bool

    
    % collect Inversion [ ]
    \dcparse_ExtractInversion:NN \l__Chord_tl \l__Inversion_tl

    % Trace
    \iow_term:n {ParseChord-Inversion}
    \tl_show:N \l__Chord_tl
    \tl_show:N \l__Inversion_tl

    % collect the Notes
    \dcparse_ExtractNotes:N \l__Chord_tl
    
}


%--------------------


% Parse a list of Note and push the result in variable
    % Prepare seq(s)
    % Parse the notes
% #1 = Notes={note-note ...}
\NewDocumentCommand\ParseNotes{m}
{
    % Temp variables
    \tl_set:Ne \l__Notes_tl {#1} % fill by Notes
    
    % Result variables at the Notes level
    \seq_clear_new:N \l__Notes_seq % List of Found Notes

    % Result variables at the Single Note level
    \seq_clear_new:N \l__CleanNote_seq % List of Node name without Pre
    \seq_clear_new:N \l__Pre_seq % List of Pre
    \seq_clear_new:N \l__Pre_bool_seq % List of Pre

    % Pretraitement for the Pre
    % ADD x before / to ease parsing 
    \regex_replace_all:nnN {\s/|^/|-/} {\0x} \l__Notes_tl

     % ADD x before ↓ & ↑ to ease parsing 
    \regex_replace_all:nnN {\s↓|^↓|-↓|\s↑|^↑|-↑} {\0x} \l__Notes_tl
   
       % Trace
        \iow_term:n {ParseNotes-After regex}
        \tl_show:N \l__Notes_tl
    
    \seq_set_split:NnV \l__Notes_seq {~} \l__Notes_tl % split by Notes

    % Parse each Note
    \seq_map_inline:Nn \l__Notes_seq 
    {
        \ParseSingleNote{##1}
        
        \seq_put_right:NV \l__CleanNote_seq \l__CleanNote_tl
        \seq_put_right:NV \l__Pre_seq \l__Pre_tl
        \seq_put_right:Ne \l__Pre_bool_seq {\bool_to_str:N \l__Pre_bool}    
    }

        % Trace
        \iow_term:n {ParseNotes-fin}
        \tl_show:N \l__Notes_tl
        \seq_show:N \l__Notes_seq
        \seq_show:N \l__CleanNote_seq
        \seq_show:N \l__Pre_seq
        \seq_show:N \l__Pre_bool_seq
}

% Parse a Note and push the result in variable
    % Detect and extract Pre   
% #1 = Notes={PrexNote}
\NewDocumentCommand\ParseSingleNote{m}
{
    % Temp variables
    \tl_set:Ne \l__Note_tl {#1} % fill by Notes

    % output variables
    \tl_clear_new:N \l__CleanNote_tl
    \tl_clear_new:N \l__Pre_tl
    \bool_set_false:N \l__Pre_bool

    % TEST and collect PRE sep = x 
    \dcparse_SplitandSearch:NnnNN \l__Note_tl  {x} {B} \l__Pre_bool \l__Pre_tl 

    \tl_set_eq:NN \l__CleanNote_tl \l__Note_tl
    
    % Trace
    \iow_term:n {ParseSingleNote-fin}
    \tl_show:N \l__CleanNote_tl
    \bool_show:N \l__Pre_bool
    \tl_show:N \l__Pre_tl

}


\ExplSyntaxOff

