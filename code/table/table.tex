\def\tabver{V185}

\usepackage{nicematrix} %Block

\usepackage{fancyhdr}

\pagestyle{fancy}
\setlength{\headheight}{15pt}
\fancyhead[L,R]{}
\fancyhead[C]{\textbf{Titre}}
\fancyfoot[C,R]{}
\fancyfoot[L]{\tiny{versions dico:\oover-\dicover \thinspace code:\codever}\newline}

%%%%
\def\tbpaver{V100}


\ExplSyntaxOn
% Usage ParsePages {file name with content}
% split each Page ||| separator
\NewDocumentCommand\ParsePages{m}
{
   % Temp variables
    % use \g_tmpa_ior to avoid error on the second Pages
    % \ior_new:N \l__Music_ior
    \tl_clear_new:N \l__Pages_tl
    \tl_clear_new:N \l__Left_tl
    
    % Result variables at the Pages level
    \seq_clear_new:N \l__Pages_seq % raw page seq
    %\tl_clear_new:N \l__PageTitle_tl % Overall Title

       
    % read file and delate \par
    \file_if_exist:nTF {#1}
        {   
        \ior_open:Nn \g_tmpa_ior { #1 }
        \ior_str_map_inline:Nn \g_tmpa_ior 
            { \tl_gput_right:Nn \l__Pages_tl {##1~} }
        \ior_close:N \g_tmpa_ior
        }
        {no file! rerun}

     % split by |||
    \seq_set_split:Nnx \l__Pages_seq {|||} {\l__Pages_tl}
    % clean if the left is empty
    \seq_get_left:NN \l__Pages_seq \l__Left_tl
    \tl_if_empty:NTF \l__Left_tl 
        {%yes empty filter
        \seq_pop_left:NN \l__Pages_seq \l__Left_tl}
        {}

        % Trace
        \iow_term:n {ParsePages-Pagesplit}
        \tl_show:N \l__Pages_tl
        \seq_show:N \l__Pages_seq  
}

% Usage ParsePage {content}
% split the Page in lines || separator
\NewDocumentCommand\ParsePage{m}
{
   % Temp variables
    \tl_clear_new:N \l__Page_tl
    \tl_set:Nn \l__Page_tl {#1}
    \tl_clear_new:N \l__Left_tl
    
    % Result variables at the Page level
    \seq_clear_new:N \l__Page_seq % raw page seq
    \tl_clear_new:N \l__PageTitle_tl % Overall Title

     % split by ||
    \seq_set_split:Nnx \l__Page_seq {||} {\l__Page_tl}
    % clean if the left is empty
    \seq_get_left:NN \l__Page_seq \l__Left_tl
    \tl_if_empty:NTF \l__Left_tl 
        {%yes empty filter
            \seq_pop_left:NN \l__Page_seq \l__Left_tl
        }
        {}
    % Title if the first is not a ligne (do not contain |)
    \seq_get_left:NN \l__Page_seq \l__Left_tl
    \tl_if_in:NnTF \l__Left_tl {|}
        {}% not a title
        {% No extract the title
            \seq_pop_left:NN \l__Page_seq \l__PageTitle_tl
        }
    
        % Trace
        \iow_term:n {ParsePage-Pagesplit}
        \tl_show:N \l__Page_tl
        \seq_show:N \l__Page_seq
        \tl_show:N \l__PageTitle_tl
}

% Usage ParseLine {content}
% split the Line in Content | separator
\NewDocumentCommand\ParseLine{m}
{
   % Temp variables
    \tl_clear_new:N \l__Line_tl
    \tl_set:Nn \l__Line_tl {#1}
    \tl_clear_new:N \l__Left_tl
    \seq_clear_new:N \l__Regex_seq
        
    % Result variables at the Line level
    \tl_clear_new:N \l__LineTitle_tl % Line Titles
    \seq_clear_new:N \l__Line_seq % raw Content seq

    % add default color
    \tl_set:Nn \l__LineColor_tl {white}

    % split by |
    \seq_set_split:Nnx \l__Line_seq {|} {\l__Line_tl}
    % clean if the left is empty (no title)
    \seq_get_left:NN \l__Line_seq \l__Left_tl
    \tl_if_empty:NTF \l__Left_tl 
        {%yes empty filter
            \seq_pop_left:NN \l__Line_seq \l__Left_tl
        }
        {% not empty T alone or with color &/or titles

            % extract and store T[color] (Tiles)
            \seq_pop_left:NN \l__Line_seq \l__LineTitle_tl
            
            % Extract Color
            \regex_extract_once:nVN {\[[^)]*\]} \l__LineTitle_tl \l__Regex_seq
            
                % Trace
                \iow_term:n {ParseLine-extract color}
                \tl_show:N \l__LineTitle_tl
                \seq_show:N \l__Regex_seq         
                       
            \seq_if_empty:NTF \l__Regex_seq
                {}% no color 
                {% yes color found
                    \tl_set:Ne \l__LineColor_tl {\seq_item:Nn \l__Regex_seq 1}

                    % delete color from l__LineTitle_tl
                    \tl_replace_once:NVn \l__LineTitle_tl \l__LineColor_tl {}
                    
                    % clean color []
                    \tl_replace_all:Nnn \l__LineColor_tl {[} {}
                    \tl_replace_all:Nnn \l__LineColor_tl {]} {}                    
                }

        }

        % Trace
        \iow_term:n {ParseLine-Linesplit}
        \tl_show:N \l__Line_tl
        \seq_show:N \l__Line_seq
        \tl_show:N \l__LineTitle_tl
        \tl_show:N \l__LineColor_tl
}

% Usage ParseContent {content}
% Get the Class
\NewDocumentCommand\ParseContent{m}
{
   % Temp variables
    \tl_clear_new:N \l__Content_tl
    \tl_set:Nn \l__Content_tl {#1}
    \tl_clear_new:N \l__Left_tl
    \seq_clear_new:N \l__Regex_seq
    
    
    % Result variables at the Page level
    \tl_clear_new:N \l__Class_tl % Content Class
    \tl_clear_new:N \l__CleanContent_tl % The content itself

     % Extract Class Before the first space (if len=1)
    \regex_extract_once:nVN {^(.*?)\s|^(.)$} \l__Content_tl \l__Regex_seq
    \seq_remove_all:Nn \l__Regex_seq {} % try to merge find 1 and Find 2
    \tl_set:Ne \l__Left_tl {\seq_item:Nn \l__Regex_seq 2}
    \tl_if_single:NTF \l__Left_tl
        {% len = 1 Class
        \tl_set_eq:NN \l__Class_tl \l__Left_tl
        \regex_replace_once:nnN {^(.*?)\s|^(.)$} {} \l__Content_tl
        \tl_set_eq:NN \l__CleanContent_tl \l__Content_tl      
        }
        {% Len <> 1 Content
        \tl_set_eq:NN \l__CleanContent_tl \l__Content_tl
        }

        % Trace
        \iow_term:n {ParseContent-parse}
        \tl_show:N \l__Content_tl
        \seq_show:N \l__Regex_seq
        \tl_show:N \l__Class_tl
        \tl_show:N \l__CleanContent_tl
}

% Usage ParsePageTitle D/H(Title)
% Get the Class
\NewDocumentCommand\ParsePageTitle{m}
{
    % Temp variables
    \tl_clear_new:N \l__PageTitle_tl
    \tl_set:Nn \l__PageTitle_tl {#1}
    \tl_clear_new:N \l__Left_tl
    \seq_clear_new:N \l__Regex_seq
    
    % Result variables at the Page level
    \bool_set_false:N \l__Half_bool % true if Double
    \tl_clear_new:N \l__CleanPageTitle_tl % Overall Title
     
     % Extract Class 
    \regex_extract_once:nVN {^(.*?)\s|^(.)$|^(.)\(} \l__PageTitle_tl \l__Regex_seq
    \seq_remove_all:Nn \l__Regex_seq {} % try to merge find 1 and Find 2 and Find 3          
    \tl_set:Ne \l__Left_tl {\seq_item:Nn \l__Regex_seq 2}

    \str_case:NnTF \l__Left_tl % not tl_case
        {
        {H}{\bool_set_true:N \l__Half_bool}
        {D}{}
        }
        {}% nothing else if found
        {}% nothing else if other
           
    % Extract Title
    \regex_extract_once:nVN {\(([^)]*)\)} \l__PageTitle_tl \l__Regex_seq
    \int_compare:nNnTF {\seq_count:N \l__Regex_seq}>{1} % test if found
        {% yes found
        \tl_set:Ne \l__CleanPageTitle_tl {\seq_item:Nn \l__Regex_seq 2 } 
        }
        {}% no        

        % Trace
        \iow_term:n {ParsePageTitle}
        \tl_show:N \l__PageTitle_tl
        \bool_show:N \l__Half_bool
        \tl_show:N \l__CleanPageTitle_tl            
}

% Usage ParseLineTitle T (Title1)..up to 4 title 
% nota: [color] has been extrated before
% Get the Class
\NewDocumentCommand\ParseLineTitle{m}
{
    % Temp variables
    \tl_clear_new:N \l__LineTitle_tl
    % to face a bug here too
    \str_set:Ne \l__LineTitle_str {#1}
    \tl_set:Ne \l__LineTitle_tl {\l__LineTitle_str}
    \tl_clear_new:N \l__Left_tl
    \tl_clear_new:N \l__Inside_tl
    \seq_clear_new:N \l__Regex_seq

    % Result variables at the Line level
    \seq_clear_new:N \l__LineTitles_seq

    % Extract Class 
    \regex_extract_once:nVN {^(.)\s|^(.)$|^(.)\(|^(.)\[} \l__LineTitle_tl \l__Regex_seq
    \seq_remove_all:Nn \l__Regex_seq {} % try to merge find 1 and Find 2 and Find 3          
    \tl_set:Ne \l__Left_tl {\seq_item:Nn \l__Regex_seq 2}

            % Trace
            \iow_term:n {ParseLineTitle-extract Class}
            \tl_show:N \l__LineTitle_tl
            \seq_show:N \l__Regex_seq
            \tl_show:N \l__Left_tl
    
    \str_case:NnTF \l__Left_tl % not tl_case
        {
        {T} % Title with name
            {
            \regex_extract_all:nVN {\([^)]*\)} \l__LineTitle_tl \l__Regex_seq
            
                % Trace
                \iow_term:n {ParseLineTitle-extract Title}
                \tl_show:N \l__LineTitle_tl
                \seq_show:N \l__Regex_seq         
                       
            \seq_if_empty:NTF \l__Regex_seq
                {}% no Title 
                {% yes Title found
                    \seq_map_inline:Nn  \l__Regex_seq
                        {
                            \tl_set:Ne \l__Inside_tl {##1}
                            % clean ()
                            \tl_replace_all:Nnn \l__Inside_tl {(} {}
                            \tl_replace_all:Nnn \l__Inside_tl {)} {}
                            % store
                            \seq_put_right:NV \l__LineTitles_seq \l__Inside_tl
                        }                   
                } 
            }
        }
        {}% nothing else if found
        {}% nothing else if other

        % Trace
        \iow_term:n {ParseLineTitle}
        \tl_show:N \l__LineTitle_tl
        \seq_show:N \l__LineTitles_seq
}

\ExplSyntaxOff
%%%%


\newcolumntype{P}[1]{>{\centering\arraybackslash}b{#1}}
\newcolumntype{N}{@{}m{0pt}@{}}


% Parse Pages in Pages / Lignes and contents
% then Draw
% Usage Pages{Name of file with content}
\ExplSyntaxOn
\NewDocumentCommand\Pages{m}
{    
    % Temp variables
    \seq_clear_new:N \l__allPageTitle_seq
    \seq_clear_new:N \l__allNbLignebyPage_seq
    
    \seq_clear_new:N \l__allLineColor_seq
    \seq_clear_new:N \l__allLineTitle_seq
    \seq_clear_new:N \l__allNbContentbyLine_seq
    
    \seq_clear_new:N \l__allClass_seq
    \seq_clear_new:N \l__allCleanContent_seq
    \seq_clear_new:N \l__allContentColor_seq

    % at the Page level
    \tl_clear_new:N \l__PageContent_tl
    \int_zero_new:N \l__LineNB_int

    % at the Line level
    \tl_clear_new:N \l__LineColor_tl
    \tl_clear_new:N \l__FullLineTitle_tl
    \int_zero_new:N \l__contentNB_int
    \int_zero_new:N \l__Toadd_int
    \tl_clear_new:N \l__contentNB_tl
    \tl_clear_new:N \l__DrawableLine_tl
    \seq_clear_new:N \l__ColumnColor_seq
    \seq_clear_new:N \l__TitleColor_seq

    % at the content level
    \tl_clear_new:N \l__CurrentClass_tl
    \tl_clear_new:N \l__ContentClass_tl
    \tl_clear_new:N \l__Content_tl
    \tl_clear_new:N \l__ContentColor_tl
    
    
    % PARSE
    
    % split the content by Page
    \ParsePages{#1} 

    % go-on if not empty
    \tl_if_blank:nTF {\l__Pages_tl}
        {}%empty
        {% not empty
         % for each Pages
            \seq_map_inline:Nn \l__Pages_seq 
            {
            % split the page by Line
            \ParsePage{##1}
            
            % store at the page level
            \seq_put_right:NV \l__allPageTitle_seq \l__PageTitle_tl
            \seq_put_right:Ne \l__allNbLignebyPage_seq {\seq_count:N \l__Page_seq}
            

            % go-on if not empty
            \tl_if_blank:nTF {\l__Page_tl}
                {}%empty
                {% not empty
                 % for each Line
                \seq_map_inline:Nn \l__Page_seq
                    {                   
                    %split the Line by Content
                    \ParseLine{####1}
                    
                    % store at the Line level
                    \seq_put_right:NV \l__allLineColor_seq \l__LineColor_tl
                    \seq_put_right:NV \l__allLineTitle_seq \l__LineTitle_tl
                    \seq_put_right:Ne \l__allNbContentbyLine_seq {\seq_count:N \l__Line_seq}
                    
                    % go-on if not empty
                    \tl_if_blank:nTF {\l__Line_tl}
                        {}%empty
                        {% not empty
                        % for each content
                        \seq_map_inline:Nn \l__Line_seq
                            {                   
                                %Parse to get Class
                                \ParseContent{########1}
                                
                                % store at the Content level
                                \seq_put_right:NV \l__allClass_seq \l__Class_tl
                                \seq_put_right:NV \l__allCleanContent_seq \l__CleanContent_tl
                                \seq_put_right:NV \l__allContentColor_seq \l__ContentColor_tl
                                
                            }
                        }
                    }
                }
            }
        }

        % Trace
        \iow_term:n {Pages-startdraw}
        \seq_show:N \l__allPageTitle_seq
        \seq_show:N \l__allNbLignebyPage_seq
    
        \seq_show:N \l__allLineColor_seq
        \seq_show:N \l__allLineTitle_seq
        \seq_show:N \l__allNbContentbyLine_seq
    
        \seq_show:N \l__allClass_seq
        \seq_show:N \l__allCleanContent_seq
        \seq_show:N \l__allContentColor_seq

    % DRAW

    % For each Page
    \seq_map_indexed_inline:Nn \l__allPageTitle_seq
    {% for each Page

        % Parse and Draw Title
        \ParsePageTitle{##2}
        \fancyhead[C]{\textbf{\l__CleanPageTitle_tl}}
        
        % Prepare the content to draw clean the content
        \tl_clear_new:N \l__PageContent_tl

        % nb of ligne to draw
        \int_set:Nn \l__LineNB_int {\seq_item:Nn \l__allNbLignebyPage_seq ##1}

        % for each line
        \int_while_do:nn {\l__LineNB_int > 0}
            {
            
            %
			% Prepare the content row of the line - Nota start to work at the content row to extract the content colors to potentially color the title row
			%
			
			% same color than title row (old spec no longer required with color in title and content)
			% \tl_put_right:Ne \l__PageContent_tl {\RowColor{\l__LineColor_tl}}
			
			\tl_clear_new:N \l__DrawableLine_tl % prepare the content to push later
			\seq_clear_new:N \l__ColumnColor_seq
			
			% consume the number of content
			% default Class = s
			\tl_set:Nn \l__CurrentClass_tl {s}
			\seq_pop_left:NN \l__allNbContentbyLine_seq \l__contentNB_tl
			\int_set:Nn \l__contentNB_int {\l__contentNB_tl}
			\int_while_do:nn {\l__contentNB_int > 0}
			{
				% read and consume the Class, Content & Color
				\seq_pop_left:NN \l__allClass_seq \l__ContentClass_tl
				\seq_pop_left:NN \l__allCleanContent_seq \l__Content_tl
				\seq_pop_left:NN \l__allContentColor_seq \l__ContentColor_tl
				
				% Push Content with Class
				\DrawContent{\l__ContentClass_tl}{\l__Content_tl}{\l__ContentColor_tl}
				\tl_put_right:Ne \l__DrawableLine_tl { \l__DrawableContent_tl}
				
				% mandatory - next content
				\int_decr:N \l__contentNB_int
			}			
                       % Trace
						\iow_term:n {Pages-prepareContent}
						\tl_show:N \l__DrawableLine_tl
						\seq_show:N \l__ColumnColor_seq

            %
            % Push the title row of the line
            %
            
                % read and consume the Color
                \seq_pop_left:NN \l__allLineColor_seq \l__LineColor_tl
                
                % color put at the title level
                %\tl_put_right:Ne \l__PageContent_tl {\RowColor{\l__LineColor_tl}}
                
                % read and consume the title
                \seq_pop_left:NN \l__allLineTitle_seq \l__FullLineTitle_tl
                % parse the LineTitle to extract the titles
                \ParseLineTitle{\l__FullLineTitle_tl}
                
                % comprehend the title to 4 if Double Table and to 2 if Half Table
                % the purpose if to have a full line color w/o rowcolor
                \bool_if:NTF \l__Half_bool
	                { \int_set:Nn \l__Toadd_int {2 - \seq_count:N \l__LineTitles_seq}}
	                { \int_set:Nn \l__Toadd_int {4 - \seq_count:N \l__LineTitles_seq}}
                                                            
                \int_while_do:nn {\l__Toadd_int > 0}
                {
                	\seq_put_right:Nn \l__LineTitles_seq {}            	
                	% mandatory next title
                	\int_decr:N \l__Toadd_int
                }
  
		                
				% Compute the Title color : either \l__LineColor_tl exept if \l__LineColor_tl="content" in that case color came from content (column 1/5/9/13)
				\seq_clear_new:N \l__TitleColor_seq
				
				\seq_map_indexed_inline:Nn \l__LineTitles_seq
					{
							\str_if_eq:VnTF \l__LineColor_tl {content}
							{%YES
								\seq_put_right:Ne \l__TitleColor_seq {\seq_item:Ne \l__ColumnColor_seq {1+4*(####1-1)}}  
							}
							{%NO
								\seq_put_right:Ne \l__TitleColor_seq {\l__LineColor_tl}    
							}
							
					}
	                      % Trace
						\iow_term:n {Pages-linetitle}
						\seq_show:N \l__LineTitles_seq
						\seq_show:N \l__TitleColor_seq				
                         
                % push the titles
                \seq_map_indexed_inline:Nn \l__LineTitles_seq
                    {
                   		\tl_put_right:Ne \l__PageContent_tl {\RowTitle{####2}{\seq_item:Nn \l__TitleColor_seq ####1}}
                    }
                % push EOL
                \tl_put_right:Ne \l__PageContent_tl {\EOL}
            
                     
                %
                % Push the content row of the line
                %    
                    
                % Push the content
            	\tl_put_right:Ne \l__PageContent_tl { \l__DrawableLine_tl}
                % Push EOL + Horiz. line
                \tl_put_right:Ne \l__PageContent_tl {\EOLHline}
                
                % mandatory - next line
                \int_decr:N \l__LineNB_int
            }
                 
        % Draw the page
        \bool_if:NTF \l__Half_bool
            {\PageH{\l__PageContent_tl}}
            {\PageD{\l__PageContent_tl}}
            
        % Trace
        \iow_term:n {Pages-end}
        \tl_show:N \l__PageContent_tl
  
    }       
}
\ExplSyntaxOff

\ExplSyntaxOn
% DrawContent with the right class {Class}{Content}{ContentColor}
\NewDocumentCommand\DrawContent{mmm}
{
	 % Temp variables
	\tl_clear_new:N \l__TheClass_tl
	\tl_set:Ne \l__TheClass_tl {#1}
	\tl_clear_new:N \l__TheContent_tl
	\tl_set:Ne \l__TheContent_tl {#2}
	\tl_clear_new:N \l__TheColor_tl
	\tl_set:Ne \l__TheColor_tl {#3}
	


	% Result variables at the Content level
	\tl_clear_new:N \l__DrawableContent_tl
	
	
	% init default Class if required
	\tl_if_empty:NTF \l__TheClass_tl
	{% empty put the Current
		\tl_set:NV  \l__TheClass_tl  \l__CurrentClass_tl
	}
	{} %not empty
	
% |O = Phrase scale =1 - 2 col
% |o = Phrase scale=.5 - 1 col
% |C = suite de Chords - 2 col
% |c = suite de Chords - 1 col
% |M = Master- 2 col
% |m = Master - 1 col
% |N = suite de Notes séparateur de block - 2 col
% |n = suite de Notes séparateur de block - 1 col
% |P[color] = suite de Notes séparateur de block scale=petit - 2 col
% |p[color] = suite de Notes séparateur de block scale=petit  - 1 col
% |S = suite de symboles (commandes latex) - 2 col
% |s = suite de symboles (commandes latex) - 1 col
% | = identique au précedant defaut s
% |1 ou 2 = une ou deux colonnes vide


	\str_case:NnTF \l__TheClass_tl % what Class
	{
	{O}
			{	\tl_set:Nn  \l__DrawableContent_tl  {\BTwo{\Phrase{\l__TheContent_tl}[1.0]}{\l__TheColor_tl}}				
			     \seq_put_right:NV \l__ColumnColor_seq \l__TheColor_tl
			     \seq_put_right:NV \l__ColumnColor_seq \l__TheColor_tl
			}
	{o}
			{	\tl_set:Nn  \l__DrawableContent_tl  {\BOne{\Phrase{\l__TheContent_tl}[0.5]}{\l__TheColor_tl}}
				 \seq_put_right:NV \l__ColumnColor_seq \l__TheColor_tl
			}
	{C}
			{	\tl_set:Nn  \l__DrawableContent_tl  {\BTwo{\Cadence{\l__TheContent_tl}[1.0]}{\l__TheColor_tl}}			
			     \seq_put_right:NV \l__ColumnColor_seq \l__TheColor_tl
				 \seq_put_right:NV \l__ColumnColor_seq \l__TheColor_tl
			}
	{c}
			{	\tl_set:Nn  \l__DrawableContent_tl  {\BOne{\Cadence{\l__TheContent_tl}[0.5]}{\l__TheColor_tl}}			
				 \seq_put_right:NV \l__ColumnColor_seq \l__TheColor_tl
			}
	{M}
			{	\tl_set:Nn  \l__DrawableContent_tl  {\BTwo{\Master{\l__TheContent_tl}[1.0]}{\l__TheColor_tl}}			
			 	\seq_put_right:NV \l__ColumnColor_seq \l__TheColor_tl
				\seq_put_right:NV \l__ColumnColor_seq \l__TheColor_tl
			}
	{m}
			{	\tl_set:Nn  \l__DrawableContent_tl  {\BOne{\Master{\l__TheContent_tl}[0.5]}{\l__TheColor_tl}}			
				 \seq_put_right:NV \l__ColumnColor_seq \l__TheColor_tl			
			}
	{N}
			{	\tl_set:Nn  \l__DrawableContent_tl  {\BTwo{\NotesBox{\l__TheContent_tl}[1.0]}{\l__TheColor_tl}}			
				 \seq_put_right:NV \l__ColumnColor_seq \l__TheColor_tl			
				 \seq_put_right:NV \l__ColumnColor_seq \l__TheColor_tl			
			}
	{n}
			{	\tl_set:Nn  \l__DrawableContent_tl  {\BOne{\NotesBox{\l__TheContent_tl}[1.0]}{\l__TheColor_tl}}			
				 \seq_put_right:NV \l__ColumnColor_seq \l__TheColor_tl			
			}
	{P}
			{	\tl_set:Nn  \l__DrawableContent_tl  {\BTwo{\NotesBox{\l__TheContent_tl}[0.5]}{\l__TheColor_tl}}			
				 \seq_put_right:NV \l__ColumnColor_seq \l__TheColor_tl
				 \seq_put_right:NV \l__ColumnColor_seq \l__TheColor_tl			
			}
	{p}
			{	\tl_set:Nn  \l__DrawableContent_tl  {\BOne{\NotesBox{\l__TheContent_tl}[0.5]}{\l__TheColor_tl}}			
				 \seq_put_right:NV \l__ColumnColor_seq \l__TheColor_tl			
			}
	{S}
			{	\tl_set:Nn  \l__DrawableContent_tl  {\BTwo{\Shapes{\l__TheContent_tl}[1.0]}{\l__TheColor_tl}}		
				 \seq_put_right:NV \l__ColumnColor_seq \l__TheColor_tl			
				 \seq_put_right:NV \l__ColumnColor_seq \l__TheColor_tl			
			}
	{s}
			{	\tl_set:Nn  \l__DrawableContent_tl  {\BOne{\Shapes{\l__TheContent_tl}[0.5]}{\l__TheColor_tl}}		
				 \seq_put_right:NV \l__ColumnColor_seq \l__TheColor_tl			
			}
	{2}
			{	\tl_set:Nn  \l__DrawableContent_tl  {\BTwo{}{\l__TheColor_tl}}		
				 \seq_put_right:NV \l__ColumnColor_seq \l__TheColor_tl			
				 \seq_put_right:NV \l__ColumnColor_seq \l__TheColor_tl			
			}
	{1}
			{	\tl_set:Nn  \l__DrawableContent_tl  {\BOne{}{\l__TheColor_tl}}		
				 \seq_put_right:NV \l__ColumnColor_seq \l__TheColor_tl			
			}
	
	}
	{% additional things if found
		\tl_set:NV  \l__CurrentClass_tl  \l__TheClass_tl
		}
	{% additional things if other
		\tl_set:Nn  \l__DrawableContent_tl  {Not Found&}
	}
	
        % Trace
		\iow_term:n {\DrawContent}
		\tl_show:N \l__TheClass_tl
		\tl_show:N \l__TheContent_tl
		\tl_show:N \l__TheContent_tl  
		\tl_show:N \l__TheColor_tl
}
\ExplSyntaxOff

\ExplSyntaxOn
% \NotesBox Draw a box of Notes 1/2 or 3 lines { Notes  - Notes} / -  = new line
\NewDocumentCommand\NotesBox{m O{1.0}}
{
	% Temp variables
	\tl_clear_new:N \l__TheNotes_tl
	\tl_set:Ne \l__TheNotes_tl {#1}
	\tl_clear_new:N \l__Scale_tl
	\tl_set:Ne \l__Scale_tl {#2}
	\seq_clear_new:N \l__Split_seq
	\tl_clear_new:N \l__NBLines_tl
	
	% split the lines
	\seq_set_split:Nnx \l__Split_seq {-} {\l__TheNotes_tl}
	\tl_set:Ne \l__NBLines_tl {\seq_count:N  \l__Split_seq }
	
	\str_case:NnTF \l__NBLines_tl % Nb ligne
	{
	{0}
			{}
	{1}
			{	\BTa{\seq_item:Nn \l__Split_seq 1 }[#2]	}
	{2}
			{	\BTb{\seq_item:Nn \l__Split_seq 1 }{\seq_item:Nn \l__Split_seq 2 }[#2]		}
	{3}
			{	\BTc{\seq_item:Nn \l__Split_seq 1 }{\seq_item:Nn \l__Split_seq 2 }{\seq_item:Nn \l__Split_seq 3 }[#2]	}					
	
	}
	{}% nothing else if found
	{Error in NotesBox} % if not Fount
	
	    % Trace
		\iow_term:n {NotesBox}
		\tl_show:N \l__TheNotes_tl
		\tl_show:N \l__Scale_tl
		\tl_show:N \l__NBLines_tl 
		
}
\ExplSyntaxOff

% Usage Page 16 col- Print a table in a page
\ExplSyntaxOn
\NewDocumentCommand\PageD{+m}
{
	% Temp variables
	\tl_clear_new:N \l__PageD_tl
	\tl_set:Ne \l__PageD_tl {#1}
	
	% Trace
	\iow_term:n {PageD}
	\tl_show:N \l__PageD_tl
	
\begin{table}[ht!]
\centering
    \newdimen\CellWidth
    \CellWidth=10ex

% for test with all cell
%    \begin{NiceTabular}{|{P{\CellWidth}}|{P{\CellWidth}}|{P{\CellWidth}}|{P{\CellWidth}}|{P{\CellWidth}}|{P{\CellWidth}}|{P{\CellWidth}}|{P{\CellWidth}}|{P{\CellWidth}}|{P{\CellWidth}}|{P{\CellWidth}}|{P{\CellWidth}}|{P{\CellWidth}}|{P{\CellWidth}}|{P{\CellWidth}}|{P{\CellWidth}}|N} 
%
    
    \begin{NiceTabular}{|*{4}{P{\CellWidth}}|*{4}{P{\CellWidth}}|*{4}{P{\CellWidth}}|*{4}{P{\CellWidth}}|N}        
    \hline
    1&2&3&4&5&6&7&8&9&10&11&12&13&14&15&16\\
    \hline
     
    \l__PageD_tl 
        
    \end{NiceTabular}
\end{table} 

\newpage

}
\ExplSyntaxOff

% Usage Page 8 col- Print a table in a page
\ExplSyntaxOn
\NewDocumentCommand\PageH{+m}
{
% Temp variables
\tl_clear_new:N \l__PageH_tl
\tl_set:Ne \l__PageH_tl {#1}

% Trace
\iow_term:n {PageH}
\tl_show:N \l__PageH_tl	
	
		
\begin{table}[ht!]
\centering
    \newdimen\CellWidth
    \CellWidth=10ex
    \begin{NiceTabular}{|*{4}{P{\CellWidth}}|*{4}{P{\CellWidth}}|N}        
    \hline
    1&2&3&4&5&6&7&8\\
    \hline 
    
	\l__PageH_tl
	
    \end{NiceTabular}
\end{table} 
\newpage
}
\ExplSyntaxOff

% Usage RowColor - set the line to a bckground color {color}
\NewDocumentCommand\RowColor{m}{\rowcolor{#1}}

% Usage RowTitle - Print the title of a row {Text}{Color}
\ExplSyntaxOn
\NewDocumentCommand\RowTitle{mm}{

\tl_if_empty:nTF {#2} 
	{% empty no specific Color
		\Block[l]{1-4}{\textbf{#1}}&&&&
	}
	{% yes specific color
		\Block[l,fill=#2]{1-4}{\textbf{#1}}&&&&
	}

}
\ExplSyntaxOff

% Usage BFour - Block of 4 Columns {Content}{Color}
\ExplSyntaxOn
\NewDocumentCommand\BFour{mm}{

\tl_if_empty:nTF {#2} 
	{% empty no specific Color
		\Block{1-4}{#1}&&&&
	}
	{% yes specific color
		\Block[fill=#2]{1-4}{#1}&&&&
	}

}
\ExplSyntaxOff

% Usage BTwo - Block of 2 Columns {Content}{Color}
\ExplSyntaxOn
\NewDocumentCommand\BTwo{mm}{

\tl_if_empty:nTF {#2} 
	{% empty no specific Color
		 \Block{1-2}{#1}&&
	}
	{% yes specific color
		\Block[fill=#2]{1-2}{#1}&&
	}
	 
}
\ExplSyntaxOff

% Usage BOne - Block of 1 Columns {Content}{Color}
\ExplSyntaxOn
\NewDocumentCommand\BOne{mm}{

\tl_if_empty:nTF {#2} 
	{% empty no specific Color
		 \Block{1-1}{#1}&
	}
	{% yes specific color
	 	\Block[fill=#2]{1-1}{#1}&
	}
}
\ExplSyntaxOff

% Usage BTa/b/c - Text Block 1,2 or 3 lignes {Content}[Content][Content]
\NewDocumentCommand\BTa{m O{1.0}}{\scalebox{#2}{\tikz[baseline=0.ex]{
    \begin{scope}[node distance=0.4ex,inner sep=0pt,node font=\large]     
        % 1 ligne
        \node[anchor=base] (M) at (0,0) {\Notes{#1}[#2]};
        \node [above=of M,anchor=south] (T) {};
        \node [below=of M,anchor=north] (B) {};
          
        \node [above=of T,anchor=south] (H) {};% extra space at TOP
        \node [below=of B,anchor=north] (L) {};% extra space at BOTTOM
    \end{scope}
    }}}


\NewDocumentCommand\BTb{mm O{1.0}}{\scalebox{#3}{\tikz[baseline=0.ex]{
			\begin{scope}[node distance=0.4ex,inner sep=0pt,node font=\large]				
				% 2 ligne
				\node[anchor=base] (M) at (0,0) {};
				\node [above=of M,anchor=south] (T) {\Notes{#1}[#3]};
				\node [below=of M,anchor=north] (B) {\Notes{#2}[#3]};
  
				\node [above=of T,anchor=south] (H) {};% extra space at TOP
				\node [below=of B,anchor=north] (L) {};% extra space at BOTTOM
			\end{scope}
}}}


\NewDocumentCommand\BTc{mmm O{1.0}}{\scalebox{#4}{\tikz[baseline=0.ex]{
			\begin{scope}[node distance=0.4ex,inner sep=0pt,node font=\large]				
				% 3 ligne
				\node[anchor=base] (M) at (0,0) {\Notes{#2}[#4]};
				\node [above=of M,anchor=south] (T) {\Notes{#1}[#4]};
				\node [below=of M,anchor=north] (B) {\Notes{#3}[#4]};

				\node [above=of T,anchor=south] (H) {};% extra space at TOP
				\node [below=of B,anchor=north] (L) {};% extra space at BOTTOM
			\end{scope}
}}}


% Usage EOL - insert \\
\NewDocumentCommand\EOL{}{\\}
% Usage EOL - insert \\
\NewDocumentCommand\EOLHline{}{\\ \hline}
