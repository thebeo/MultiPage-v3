\def\dicover{V496}

\usepackage{tikz} % dessin
\usetikzlibrary{arrows.meta,bending,shapes.geometric,shapes.misc,positioning}
\usepackage{xparse} % for NewDocumentCommand

%%%%
\def\dcpaver{V150}

\ExplSyntaxOn

% Parse a list of Note and push the result in variable
    % Prepare seq(s)
    % Parse the notes
% #1 = Notes={note-note ...}
\NewDocumentCommand\ParseNotes{m}
{
    % Temp variables
    \tl_set:Ne \l__Notes_tl {#1} % fill by Notes
    
    % Result variables at the Notes level
    \seq_clear_new:N \l__Notes_seq % List of Found Notes

    % Result variables at the Single Note level
    \seq_clear_new:N \l__CleanNote_seq % List of Node name without Pre
    \seq_clear_new:N \l__Pre_seq % List of Pre
    \seq_clear_new:N \l__Pre_bool_seq % List of Pre

    % Pretraitement for the Pre
    % ADD x before / to ease parsing 
    \regex_replace_all:nnN {\s/|^/|-/} {\0x} \l__Notes_tl

     % ADD x before ↓ & ↑ to ease parsing 
    \regex_replace_all:nnN {\s↓|^↓|-↓|\s↑|^↑|-↑} {\0x} \l__Notes_tl
   
       % Trace
        \iow_term:n {ParseNotes-After regex}
        \tl_show:N \l__Notes_tl
    
    \seq_set_split:NnV \l__Notes_seq {~} \l__Notes_tl % split by Notes

    % Parse each Note
    \seq_map_inline:Nn \l__Notes_seq 
    {
        \ParseSingleNote{##1}
        
        \seq_put_right:NV \l__CleanNote_seq \l__CleanNote_tl
        \seq_put_right:NV \l__Pre_seq \l__Pre_tl
        \seq_put_right:Ne \l__Pre_bool_seq {\bool_to_str:N \l__Pre_bool}    
    }

        % Trace
        \iow_term:n {ParseNotes-fin}
        \tl_show:N \l__Notes_tl
        \seq_show:N \l__Notes_seq
        \seq_show:N \l__CleanNote_seq
        \seq_show:N \l__Pre_seq
        \seq_show:N \l__Pre_bool_seq
}

% Parse a Note and push the result in variable
    % Detect and extract Pre   
% #1 = Notes={PrexNote}
\NewDocumentCommand\ParseSingleNote{m}
{
    % Temp variables
    \tl_set:Ne \l__Note_tl {#1} % fill by Notes

    % output variables
    \tl_clear_new:N \l__CleanNote_tl
    \tl_clear_new:N \l__Pre_tl
    \bool_set_false:N \l__Pre_bool

    % TEST and collect PRE sep = x 
    \dcparse_SplitandSearch:NnnNN \l__Note_tl  {x} {B} \l__Pre_bool \l__Pre_tl 

    \tl_set_eq:NN \l__CleanNote_tl \l__Note_tl
    
    % Trace
    \iow_term:n {ParseSingleNote-fin}
    \tl_show:N \l__CleanNote_tl
    \bool_show:N \l__Pre_bool
    \tl_show:N \l__Pre_tl

}


\ExplSyntaxOff


%%%%

%%%%%%
% Notes 
%%%%%%

%usage \Notes{note Note ...Note}[scale]

\ExplSyntaxOn
\NewDocumentCommand\Notes{m O{1.0}}
{
    \ParseNotes{#1}       
    \dcDrawNotes[#2]   
}
\ExplSyntaxOff

\ExplSyntaxOn
%usage \dcDrawNotes[scale]
% draw the notes in \l__CleanNote_seq
 \NewDocumentCommand\dcDrawNotes{O{1.0}}
{
    \tl_set_eq:NN \l__Size_tl \text
    \fp_zero_new:N \l__Scale_fp
    \fp_set:Nn \l__Scale_fp {#1}
        

	% Huge in both case so unusefull
    \fp_compare:nNnTF {\l__Scale_fp } = { 1 }
        {\tl_set_eq:NN \l__Size_tl \Huge}
        {}
    \fp_compare:nNnTF {\l__Scale_fp } = { .5 }
        {\tl_set_eq:NN \l__Size_tl \Huge}
        {}
 
         % Trace
		 \iow_term:n {dcDrawNotes}
		 \iow_term:n {#1}
		 \fp_show:N \l__Scale_fp
		 \tl_show:N   \l__Size_tl
    
    % Draw each Note
    \seq_map_indexed_inline:Nn \l__CleanNote_seq  
    {    
        %Draw Pre & Note
        % constante in fact to test bool
        \str_set:Nn \l__true_str {true}
        
        \str_set:Ne \l__Pre_bool_str {\seq_item:Nn \l__Pre_bool_seq {##1}}
           
        \int_compare:nNnTF {##1}>{1} % not the first one
        {% not the first draw the -
            \str_if_eq:NNTF \l__Pre_bool_str \l__true_str
            {% yes Pre
                \tl_set:Ne \l__Pre_tl {\seq_item:Nn \l__Pre_seq {##1}}
                 % If not exist replace by the unkwow note
                \cs_if_exist:cTF{dc##2}
                {-\dcPreDraw{\l__Pre_tl}[#1]\use:c {dc##2}[\l__Size_tl]}
                {-\dcPreDraw{\l__Pre_tl}[#1]\use:c {dcUKw}[\l__Size_tl]} 
            }
            {% No Pre
                 % If not exist replace by the unkwow note
                \cs_if_exist:cTF{dc##2}
                {-\use:c {dc##2}[\l__Size_tl]}
                {-\use:c {dcUKw}[\l__Size_tl]}              
            }
        }
        {% the first
            \str_if_eq:NNTF \l__Pre_bool_str \l__true_str
            {% yes Pre
                \tl_set:Ne \l__Pre_tl {\seq_item:Nn \l__Pre_seq {##1}}
                 % If not exist replace by the unkwow note
                \cs_if_exist:cTF{dc##2}
                {\dcPreDraw{\l__Pre_tl}[#1]\use:c {dc##2}[\l__Size_tl]}
                {\dcPreDraw{\l__Pre_tl}[#1]\use:c {dcUKw}[\l__Size_tl]}            }
            {% No Pre 
                 % If not exist replace by the unkwow note
                \cs_if_exist:cTF{dc##2}
                {\use:c {dc##2}[\l__Size_tl]}
                {\use:c {dcUKw}[\l__Size_tl]}
            }
        }

    }
}
\ExplSyntaxOff

% Usage dcPreDraw: Draw Pre {Pre}
\ExplSyntaxOn
\NewDocumentCommand\dcPreDraw{m O{1.0}}{     
        % Trace
        \iow_term:n {dcPreDraw}
        \iow_term:n {#1}
           
  \str_clear_new:N \l__Pre_str
  \str_set:Ne \l__Pre_str {#1}
  
  \str_case:NnTF \l__Pre_str % what gamme
 	 {

 		{/} %flame
 		{\FL{black}[][#2]}
 		{↓} %Down arrow
 		{\DA{black}[#2]}
 		{↑} %Up arrow
 		{\UA{black}[#2]}
 		{⭥} %Up arrow
 		{ \UDA{black}[#2]}
 	}
 	{}% additional things if found
 	{#1x}% additional things if other

}
\ExplSyntaxOff


% usage: \note{color}{texte}{gamme}  9=no deco
\ExplSyntaxOn
\NewDocumentCommand\note{mmm}
{
    \tl_clear_new:N \l__note_tl
    \tl_set:Nn \l__note_tl {\textcolor{#1}{\textbf{#2}}}    

    \str_clear_new:N \l__gamme_str
    \str_set:Ne \l__gamme_str {#3}

	\str_case:NnTF \l__gamme_str % what gamme
	{
		{-2} 	{ \DDC	{\l__note_tl} }
		{-1} 	{ \DC	{\l__note_tl} }
		{0}     { \UL	{\l__note_tl} }
		{1} 	{ \UC	{\l__note_tl} }
        {+1} 	{ \UC	{\l__note_tl} }
		{2} 	{ \DUC	{\l__note_tl} }
        {+2} 	{ \DUC	{\l__note_tl} }
		{9} 	{        \l__note_tl  }
	}
	{}% additional things if found
	{}% additional things if other
}
\ExplSyntaxOff


\ExplSyntaxOn
% usage: \do [gamme]
\NewDocumentCommand\DO{O{9}}{\text{\note{do}{do}{#1}}}
% [size] \text to do nothing

\NewDocumentCommand\dcDOb{O{\text}}
{	 
	% Trace
	\tl_set:Nn \l__PAR_tl  {#1}
	\iow_term:n {dcDOb}
	\tl_show:N \l__PAR_tl
	
	{#1{\DO[-1]}}
}
\NewDocumentCommand\dcDOu{O{\text}}{{#1{\DO[+1]}}}
\NewDocumentCommand\dcXDOb{O{\text}}{\CE{do}}
\NewDocumentCommand\dcXDOu{O{\text}}{\CE{do}}
% usage: \re 
\NewDocumentCommand\RE{O{9}}{\text{\note{re}{ré}{#1}}}

\NewDocumentCommand\dcREd{O{\text}}{{#1{\RE[-2]}}}
\NewDocumentCommand\dcREm{O{\text}}{{#1{\RE[0]}}}
\NewDocumentCommand\dcREh{O{\text}}{{#1{\RE[+2]}}}
\NewDocumentCommand\dcXREd{O{\text}}{\CE{re}}
\NewDocumentCommand\dcXREm{O{\text}}{\CE{re}}
\NewDocumentCommand\dcXREh{O{\text}}{\CE{re}}
% usage: \mi 
\NewDocumentCommand\MI{O{9}}{\text{\note{mi}{mi}{#1}}}

\NewDocumentCommand\dcMIl{O{\text}}{{#1{\MI[-2]}}}
\NewDocumentCommand\dcMIm{O{\text}}{{#1{\MI[0]}}}
\NewDocumentCommand\dcMIh{O{\text}}{{#1{\MI[+2]}}}
\NewDocumentCommand\dcXMIl{O{\text}}{\CE{mi}}
\NewDocumentCommand\dcXMIm{O{\text}}{\CE{mi}}
\NewDocumentCommand\dcXMIh{O{\text}}{\CE{mi}}
% usage: \fa 
\NewDocumentCommand\FA{O{9}}{\text{\note{fa}{fa}{#1}}}

\NewDocumentCommand\dcFAl{O{\text}}{{#1{\FA[-2]}}}
\NewDocumentCommand\dcFAu{O{\text}}{{#1{\FA[+1]}}}
\NewDocumentCommand\dcFAh{O{\text}}{{#1{\FA[+2]}}}
\NewDocumentCommand\dcXFAl{O{\text}}{\CE{fa}}
\NewDocumentCommand\dcXFAu{O{\text}}{\CE{fa}}
\NewDocumentCommand\dcXFAh{O{\text}}{\CE{fa}}
% usage: \sol 
\NewDocumentCommand\SOL{O{9}}{\text{\note{sol}{sol}{#1}}}

\NewDocumentCommand\dcSOLl{O{\text}}{{#1{\SOL[-2]}}}
\NewDocumentCommand\dcSOLu{O{\text}}{{#1{\SOL[+1]}}}
\NewDocumentCommand\dcSOLh{O{\text}}{{#1{\SOL[+2]}}}
\NewDocumentCommand\dcXSOLl{O{\text}}{\CE{sol}}
\NewDocumentCommand\dcXSOLu{O{\text}}{\CE{sol}}
\NewDocumentCommand\dcXSOLh{O{\text}}{\CE{sol}}
% usage: \la 
\NewDocumentCommand\LA{O{9}}{\text{\note{la}{la}{#1}}}

\NewDocumentCommand\dcLAb{O{\text}}{{#1{\LA[-1]}}}
\NewDocumentCommand\dcLAu{O{\text}}{{#1{\LA[+1]}}}
\NewDocumentCommand\dcLAh{O{\text}}{{#1{\LA[+2]}}}
\NewDocumentCommand\dcXLAb{O{\text}}{\CE{la}}
\NewDocumentCommand\dcXLAu{O{\text}}{\CE{la}}
\NewDocumentCommand\dcXLAh{O{\text}}{\CE{la}}
% usage: \si 
\NewDocumentCommand\SI{O{9}}{\text{\note{si}{si}{#1}}}

\NewDocumentCommand\dcSIb{O{\text}}{{#1{\SI[-1]}}}
\NewDocumentCommand\dcXSIb{O{\text}}{\CE{si}}

% usage: \uk unknown note 
\NewDocumentCommand\UK{O{9}}{\text{\note{uk}{uk}{#1}}}

\NewDocumentCommand\dcUKw{O{\text}}{{#1{\UK[0]}}}
\NewDocumentCommand\dcXUKw{O{\text}}{\CE{uk}}

\ExplSyntaxOff

%%%%%%
% Octave notation
%%%%%%

% size
\newdimen\sep
\sep=.05ex
\newdimen\depth
\depth=.3ex
\newdimen\betw
\betw=.4ex
\newdimen\linwd
\linwd=.5pt

% \DC - Arc Down
% usage: \DC{text}[scale] : \DC {re-mi} simple arc bellow
\NewDocumentCommand\DC{m O{1.0}}{\scalebox{#2}{\tikz[baseline=(arced node.base)]{
        \node [inner sep=0pt, outer sep=0pt] (arced node) {#1};
        \draw [line width=\linwd] [transform canvas={yshift=-\sep}] (arced node.south west) parabola [parabola height=-\depth] (arced node.south east);
    }}}

% \DL - Ligne Down
% usage: \DL{text}[scale] : \DC{re-mi} simple lign bellow
\NewDocumentCommand\DL{m O{1.0}}{\scalebox{#2}{\tikz[baseline=(arced node.base)]{
        \node [inner sep=0pt, outer sep=0pt] (arced node) {#1};
        \draw [line width=\linwd] [transform canvas={yshift=-\sep}] (arced node.south west) -- (arced node.south east);
    }}}

% \DDC - double Arc Down
% usage: \DDC{text}[scale]
\NewDocumentCommand\DDC{m O{1.0}}{\scalebox{#2}{\tikz[baseline=(arced node.base)]{
        \node [inner sep=0pt, outer sep=0pt] (arced node) {#1};
        \draw [line width=\linwd] [transform canvas={yshift=-\sep}] (arced node.south west) parabola [parabola height=-\depth] (arced node.south east);
        \draw [line width=\linwd] [transform canvas={yshift=-\sep -\betw }] (arced node.south west) parabola [parabola height=-\depth] (arced node.south east);
    }}}

%\DDL - double ligne Down
% usage: \DUL{text}[scale] 
\NewDocumentCommand\DDL{m O{1.0}}{\scalebox{#2}{\tikz[baseline=(arced node.base)]{
        \node [inner sep=0pt, outer sep=0pt] (arced node) {#1};
        \draw [line width=\linwd] [transform canvas={yshift=-\sep}] (arced node.south west) -- (arced node.south east);
        \draw [line width=\linwd] [transform canvas={yshift=-\sep -\betw }] (arced node.south west) -- (arced node.south east);
    }}}

% \UC - Arc Up
% usage: \UC{text}[scale] : \UC{re-mi}
\NewDocumentCommand\UC{m O{1.0}}{\scalebox{#2}{\tikz[baseline=(arced node.base)]{
        \node [inner sep=0pt, outer sep=0pt] (arced node) {#1};
        \draw [line width=\linwd] [transform canvas={yshift=-\sep}] (arced node.north west) parabola [parabola height=+\depth] (arced node.north east);
    }}}

% \UL - Ligne UP
% usage: \UL{text}[scale] : \UC{re-mi}
\NewDocumentCommand\UL{m O{1.0}}{\scalebox{#2}{\tikz[baseline=(arced node.base)]{
        \node [inner sep=0pt, outer sep=0pt] (arced node) {#1};
        \draw [line width=\linwd] [transform canvas={yshift=-\sep}] (arced node.north west) -- (arced node.north east);
    }}}

% \DUC - Double Arc UP
% usage: \DUC{text}[scale] : \DUC{re-mi}
\NewDocumentCommand\DUC{m O{1.0}}{\scalebox{#2}{\tikz[baseline=(arced node.base)]{
        \node [inner sep=0pt, outer sep=0pt] (arced node) {#1};
        \draw [line width=\linwd] [transform canvas={yshift=\sep}] (arced node.north west) parabola [parabola height=+\depth] (arced node.north east);
        \draw [line width=\linwd] [transform canvas={yshift=\sep +\betw}] (arced node.north west) parabola [parabola height=+\depth] (arced node.north east);
    }}}

% \DUL - Double Ligne UP
% usage: \DUL{text}[scale] : \DUL{re-mi}
\NewDocumentCommand\DUL{m O{1.0}}{\scalebox{#2}{\tikz[baseline=(arced node.base)]{
        \node [inner sep=0pt, outer sep=0pt] (arced node) {#1};
        \draw [line width=\linwd] [transform canvas={yshift=\sep}] (arced node.north west) -- (arced node.north east);
        \draw [line width=\linwd] [transform canvas={yshift=\sep +\betw}] (arced node.north west) -- (arced node.north east);
    }}}



